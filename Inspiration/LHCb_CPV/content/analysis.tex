\section{Analysis Strategy}
\label{sec:Analysis-Strategy}
This analysis uses two data sets.
One contains simulated events of the decay \mbox{$B^{\pm} \to \symup{K}^{\pm} \symup{K}^{+} \symup{K}^{-}$} while the other are measurements from the LHCb experiment of the year 2011 with an integrated luminosity of $\SI{434}{pb}^{-1}$ for the up polarity of the magnet and $\SI{584}{pb}^{-1}$ for the down polarity.
These measurements were selected on the same decay as the simulated data.
After preselecting the data set by applying cuts in particle identification variables, global $C\!P$ violation in this data set is investigated.
By cutting on their respective masses, charmonium resonances are being removed by using a Dalitz plot.
At the end studies on local $C\!P$ violation are made by using the remaining data.



\section{Analysis}
\subsection{Invariant mass of simulated $B$ mesons}
The mass of the $B$ mesons cannot be determined directly.
Therefore it is calculated via the final state particles in the given simulated data set.
In the case of the decay
\begin{equation*}
  B^{\pm} \to \symup{K}^{\pm} \symup{K}^{+} \symup{K}^{-}\,.
\end{equation*}
the $B$ mass is given by the mass and momenta of the three kaons with the relativistic relation of
\begin{equation}
  \symup{E}^2 = \symup{p}^2 + \symup{m}^2\,.
  \label{eqn:relativityEQ}
\end{equation}
The distribution of the three momentum components of one kaon is shown in figure \ref{fig:KP}.
The x and y components have a very similar distribution due to symmetrical reasons since both are transversal components of the momentum.
The z component of the kaon momentum has a different distrubution since it is the longitudinal component directing in the direction of the proton-proton beam.
\begin{figure}[!htb]
  \centering
  \includegraphics[width=0.8\textwidth]{plots/SIM_H1_P.pdf}
  \caption{The three components x, y and z of the momentof of the first kaon candidate.}
  \label{fig:KP}
\end{figure}
To calculate the magnitude of the momentum the relation
\begin{equation}
  p = \sqrt{p_x^2+p_y^2+p_z^2} \label{eqn:pxyz}
\end{equation}
is used.
Together with the kaon mass of $\SI{493.677}{MeV}$ \cite{pdg} the energy of the kaons is calculated using (\ref{eqn:relativityEQ}).
The resulting energy distribution of the first kaon candidate is shown in figure \ref{fig:KE}.
\begin{figure}[!htb]
  \centering
  \includegraphics[width=0.8\textwidth]{plots/SIM_H1_E.pdf}
  \caption{Energy distribution calculated from the mass and momentum of the first kaon candidate.}
  \label{fig:KE}
\end{figure}
Because of conservation of energy the energy of the $B$ meson is equal to the sum of the energies of the three kaons.
The components of the momentum of the $B$ meson are calculated like
\begin{equation}
  p_i(B)=p_i(K_1)+p_i(K_2)+p_i(K_3),
\end{equation}
where $i$ can be x, y or z indicating the component of the momentum and $K_1, K_2$ and $K_3$ are the three kaons.
To get the magnitude of momentum (\ref{eqn:pxyz}) is used again.

By using the relation (\ref{eqn:relativityEQ}), this leads to an invariant mass distribution of the $B$ meson as shown in figure \ref{fig:BM}.
\begin{figure}[!htb]
  \centering
  \includegraphics[width=0.8\textwidth]{plots/SIM_B_M.pdf}
  \caption{Invariant mass distribution of the simulated $B$ mesons.}
  \label{fig:BM}
\end{figure}
The figure shows a sharp peak at the location of the known $B$ mass of $\SI{5279.26}{MeV}$ \cite{pdg}.

\subsection{Invariant mass of measured $B$ mesons}
Before calculating the invariant mass of the measured $B$ mesons, a preselection is applied.
The following cuts are taken on the given data:
\begin{enumerate}
  \item \texttt{H$_n$\_isMuon} == False
  \item \texttt{H$_n$\_ProbPi} < 0.5
  \item \texttt{H$_n$\_ProbK} > 0.5
\end{enumerate}
In this case $n$ can be 1, 2 or 3 indicating that the cuts are applied on all three final state hadrons.
The boolean \texttt{isMuon} indicates if the identified particle is classified as a muon.
\texttt{ProbPi} and \texttt{ProbK} show the probability of the hadron being a pion or a kaon.
These variables are shown for the first hadron in figure \ref{fig:ProbPi} and \ref{fig:ProbK}.
The variables of the other hadrons possess a similar distribution.

\begin{figure}[!htb]
\centering
\begin{subfigure}{0.49\textwidth}
  \includegraphics[width=\textwidth]{plots/DATA_H1_ProbPi.pdf}
  \caption{ProbPi with a cut at ProbPi < 0.5}
  \label{fig:ProbPi}
\end{subfigure}
\begin{subfigure}{0.49\textwidth}
  \includegraphics[width=\textwidth]{plots/DATA_H1_ProbK.pdf}
  \caption{ProbK with a cut at ProbK > 0.5}
  \label{fig:ProbK}
\end{subfigure}
\caption{ProbPi and ProbK for the first kaon candidate.}
\end{figure}

The invariant mass of the $B$ meson is calculated analog to the previous section.
This leads to the distribution shown in figure \ref{fig:BMData}.
\begin{figure}[!htb]
  \centering
  \includegraphics[width=0.8\textwidth]{plots/DATA_B_M.pdf}
  \caption{Invariant mass distribution of the measured $B$ mesons.}
  \label{fig:BMData}
\end{figure}
It shows some differences to the simulated data set.
On the lower and higher mass side there are some background events showing up.
Beside that the mass peak is more smeared out due to energy fluctuations and resolution effects.
Since changing the cuts on the ProbPi and ProbK variables seem to cut away a lot of events in the signal region while leaving most of the background events in the mass side bands, these variable cuts are not changed in any way.
For the further analysis cuts on the $B$ mass are being made since the mass side bands mostly contain combinatorial and partial reconstructed background.
The following mass region will be analyzed in the following:
\begin{equation*}
  5200 < B\_M < 5350 \, ,
\end{equation*}
where \texttt{B\_M} is the reconstructed mass in MeV.
The remaining events are shown in figure \ref{fig:BMSignal}.
\begin{figure}[!htb]
  \centering
  \includegraphics[width=0.8\textwidth]{plots/DATA_B_M_CUT.pdf}
  \caption{Invariant mass distribution of the measured $B$ mesons with cuts on the $B$ mass.}
  \label{fig:BMSignal}
\end{figure}

\subsection{Search for global $C\!P$ violation}
To measure $C\!P$ violation in general, the number of decays with $B^+$ ($N^+$) and with $B^-$ ($N^-$) has to be known.
To get the charge of a $B$ meson, the charge of the final states kaons has to be taken into account:
\begin{equation}
  Q(B) = Q(K_1) + Q(K_2) + Q(K_3)
\end{equation}
where $Q$ is the charge of the corresponding particle with a value of $\pm 1$.
After getting ($N^+$) and ($N^-$) by counting the remaining events, the global $C\!P$ asymmetry is calculated by using
\begin{equation}
  A = \frac{N^+-N^-}{N^++N^-} = -0.0502. \label{eqn:ACP}
\end{equation}
The corresponding statistical uncertainty and the statistical significance have values of
\begin{align}
  \label{eqn:sigmaCP}
  \sigma_A &= \sqrt{\frac{1 - A^2}{N^{+} + N^{-}} } \approx 0.0077 \\
  \label{eqn:signifikanzCP}
  \text{significance} &= \frac{\symup{A}}{\sigma_A} \approx 6.5035 \, .
\end{align}
The sicnificance of $6.5\,\sigma$ would usually mean a discovery of $C\!P$ violation.
However in this case no systematic uncertainies are being considered yet.
By considering a production asymmetry of $1\,\%$ the corrected significance has just a value of $3.978\,\sigma$, having a value smaller that $5\,\sigma$ so it can only be considered as an evidence of $C\!P$ violation.

\subsection{Dalitz plots and two-body resonances}
An important method used in three-body decays are Dalitz plots.
The decay \mbox{$B^{\pm} \to \symup{K}^{\pm} \symup{K}^{+} \symup{K}^{-}$} can be a direct decay or via $B^{\pm} \to \symup{K}^{+} \symup{R}^{0}$ where $R^0$ is an uncharged particle decaying into $K^+ K^-$.
Since the measured kaons are enumerated, there are three combinations of final state particles with two-body resonances:
\begin{enumerate}
  \item $R^{0}_1 \to K_1^+ K_2^-$
  \item $R^{++}_2 \to K_1^+ K_3^+$
  \item $R^{0}_3 \to K_3^+ K_2^-$
\end{enumerate}
The double charged resonance $R^{++}$ is not expected.
The charge of a $q\bar{q}$ pair cannot be added to a value of 2.
This leads to two remaining resonances: $R^0_1$ and $R^0_3$.
These $R^0$ resonances can be shown in a Dalitz plot.

The momenta and energies of the three final state kaons are not independent since these are conserved quantities.
By plotting the invariant mass squares of just two of the three kaons, it is possible to identify resonances occuring in the data.
Figure \ref{fig:DalitzSim} shows a Dalitz plot for the simulated data.
The plot shows an even distribution of the kinematicaly allowed region of energies and momenta.
The Dalitz plot of the measured data is shown in figure \ref{fig:DalitzData}.
Two resonance lines are recognizable.
To highlight these lines even more the resonance masses are being sorted.
The invariant masses $m_{12}$ and $m_{23}$ are being compared and the bigger mass is sorted into $R^0_\text{max}$ and the lower mass into $R^0_\text{min}$.
By using these as Dalitz variables the Dalitz plot now shows a more compromised kinetical area with more visible lines due to a higher event density.

\begin{figure}[!htb]
\centering
\begin{subfigure}{0.49\textwidth}
  \includegraphics[width=\textwidth]{plots/SIM_DALITZ.pdf}
  \caption{Dalitz plot of the simulated $B^{\pm} \to \symup{K}^{\pm} \symup{K}^{+} \symup{K}^{-}$ data.}
  \label{fig:DalitzSim}
\end{subfigure}
\begin{subfigure}{0.49\textwidth}
  \includegraphics[width=\textwidth]{plots/DATA_DALITZ.pdf}
  \caption{Dalitz plot of the measured data containing resonances.}
  \label{fig:DalitzData}
\end{subfigure}
\caption{Dalitz plots of simulated and measured data.}
\end{figure}
This is shown in figure \ref{fig:Sorted}.

\begin{figure}[!htb]
  \centering
  \includegraphics[width=0.5\textwidth]{plots/DATA_DALITZ_SORTED.pdf}
  \caption{Dalitz Plot with $R^2_\text{max}$ and $R^2_\text{min}$ as Dalitz variables.}
  \label{fig:Sorted}
\end{figure}

A binned version of the Dalitz plot is being produced to highlight the resonance lines in.
The binned Dalitz plot is shown in figure \ref{fig:DalitzBinned}
\begin{figure}[!htb]
  \centering
  \includegraphics[width=0.5\textwidth]{plots/DATA_DALITZ_BINNED.pdf}
  \caption{Binned Dalitz plot of measured data showing resonance lines.}
  \label{fig:DalitzBinned}
\end{figure}
The first visible line with an invariant mass squared of approximately $\SI{1.2e7}{MeV\squared}$ leads to a mass of $M \approx \SI{3464}{MeV}$.
The particle closest to this mass is the $\chi_{c0}(1P)$ with a mass of $\SI{3414}{MeV}$ which is decaying into $K^+K^-$ with a branching fraction of $(6.05\pm0.31) \times 10^{-3}$\cite{pdg}.
The reason for the discrepancy of the $\chi_{c0}(1P)$ mass and the resonance line is that the position of the line is just estimated and the events in the line do not form a perfect line but a distribution.
The second resonance is at an invariant mass squared of approximatelly $\SI{0.3e7}{MeV\squared}$.
The corresponding resonance mass is $M \approx \SI{1732}{MeV}$.
The particle this mass belongs to is the $D^0$ with a mass of $\SI{1864}{MeV}$ decaying into $K^+K^-$ with a branching fraction of $(4.08\pm0.06) \times 10^{-3}$\cite{pdg}.
Another particle which is in this mass region decaying into $K^+K^-$ is the $J/\psi$ with a mass of $\SI{3096}{MeV}$.
Even though the mass is not directly on the resonance lines of the Dalitz plots, it is still being taken into account.
After cutting on the $\chi_{c0}(1P)$, $D^0$ and $J/\psi$ masses with the cuts
\begin{align*}
 \SI{1800}{\mega\electronvolt} < B_M < \SI{1900}{\mega\electronvolt}  \\
 \SI{3080}{\mega\electronvolt} < B_M < \SI{3120}{\mega\electronvolt} \\
 \SI{3380}{\mega\electronvolt} < B_M < \SI{3450}{\mega\electronvolt}
\end{align*}
the two Dalitz plots from $B^+$ and $B^-$ decays are being compared.
These are shown in figure \ref{fig:DalitzBPLUS} and \ref{fig:DalitzBMINUS}.

\begin{figure}[!htb]
\centering
\begin{subfigure}{0.49\textwidth}
  \includegraphics[width=\textwidth]{plots/DATA_DALITZ_BPLUS.pdf}
  \caption{Binned Dalitz plot of $B^+$.}
  \label{fig:DalitzBPLUS}
\end{subfigure}
\begin{subfigure}{0.49\textwidth}
  \includegraphics[width=\textwidth]{plots/DATA_DALITZ_BMINUS.pdf}
  \caption{Binned Dalitz plot of $B^-$.}
  \label{fig:DalitzBMINUS}
\end{subfigure}
\caption{Binned Dalitz plots of both $B^\pm$ decays after cutting out the resonances.}
\end{figure}

Since both plots look very similar, an asymmetry plot with bigger binning is made as well as a plot to show the uncertainties and the significance.
This also allows to calculate the binned asymmetries.
These plots are shown in figure \ref{fig:DalitzLocal}.

\begin{figure}[!htb]
\centering
\begin{subfigure}{0.49\textwidth}
  \centering
  \includegraphics[width=\textwidth]{plots/DATA_DALITZ_LOCAL.pdf}
  \caption{Binned Dalitz plot showing the asymetry of $B^+$ and $B^-$.}

\end{subfigure}
\begin{subfigure}{0.49\textwidth}
  \centering
  \includegraphics[width=\textwidth]{plots/DATA_DALITZ_LOCAL_UNCERTAINTY.pdf}
  \caption{Uncertainty of the asymetry.}

\end{subfigure}
\begin{subfigure}{0.49\textwidth}
  \centering
  \includegraphics[width=\textwidth]{plots/DATA_DALITZ_LOCAL_SIGNIFICANCE.pdf}
  \caption{Significance of local $C\!P$ violation}
  \label{fig:DalitzSignificance}

\end{subfigure}

\caption{Asymmetry, uncertainty and significane comparing $B^+$ and $B^-$ decays.}
\label{fig:DalitzLocal}
\end{figure}

In figure \ref{fig:DalitzSignificance} the region
\begin{align*}
 \SI{0.6e7}{\mega\electronvolt\squared} < R_\text{max}^2 < \SI{1.3e7}{\mega\electronvolt\squared}\\
 \SI{0.1e7}{\mega\electronvolt\squared} < R_\text{min}^2 < \SI{0.3e7}{\mega\electronvolt\squared}
\end{align*}
shows the largest significance.
The invariant mass distribution of $B^+$ and $B^-$ decays in that region is shown in figure \ref{fig:BPM}.
It shows the difference in $B^+$ and $B^-$ events indicating local $C\!P$ violation.

\begin{figure}[!htb]
  \centering
  \includegraphics[width=0.8\textwidth]{plots/DATA_CP.pdf}
  \caption{Comparison of $B^+$ and $B^-$ events showing the local $C\!P$ violation.}
  \label{fig:BPM}
\end{figure}

After calculating the statistical uncertainty and significance, this leads to a value of
\begin{equation*}
  A_\text{local}= -0.154 \pm 0.017 (\text{stat.})
\end{equation*}
This results in a significance of $8.99\,\sigma$.
After taking the systematic uncertainty of $1\,\%$ into account caused by production asymmetry there is still a significance of $7.77\,\sigma$.
This is an evidence of local $C\!P$ violation in the decay $B^{\pm} \to \symup{K}^{\pm} \symup{K}^{+} \symup{K}^{-}$.

\section{Discussion}
\label{sec:Diskussion}
The trained BDT showed no signs of overtraining while being very good in separating signal and background.
The ROC curve showed a good relation of tpr against fpr.
After the BDT classification, almost no background events seem to be inside the upper mass sideband.
This indicates a well trained BDT for reducing combinatorial background.


Fitting the MC of the signal and control channel seemed to work very fine.
The errors on these fit parameters are small, so this step was not problematic.
The fit of the full model was good as well, indicating no errors were made during this step.

With a signifcance of $5 \sigma$ the decay $B^0_s \rightarrow \psi(2S)K^0_S$ was observed.
Further steps in the event selection like training a second BDT to reduce partial reconstructed background events in the lower mass sideband or cutting on particle identification (PID) variables could result in even higher significances.

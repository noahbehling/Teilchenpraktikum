\section{Discussion}
\label{sec:Diskussion}

The $\theta^2$-plot for on- and off-regions shows a clear difference between the number of detected on- and off-events for angles from approximately $\theta^2 = 0$ to $\theta^2 = 0.025$. Because of this, a cut off at $0.025$ is chosen. For bigger $\theta$-values, the on-events are not distinguishable from the off-events.

Furthermore, it shows that most on-events per bin are detected around a $\theta^2$-value of $0$. This is because the on-region is directed directly at the gamma source. In contrast, the number of off-events is evenly distributed over all angles. They are background events.

The calculated significance of the detection is $S = 26.27587 \sigma$.

%It is interpreted as the probability that the assumption that there is no real gamma source conflicts with the experimental results. On the one hand, for the observed region in the crab nebula this means a relative low chance of a detected gamma source. One the other hand, a detected source cannot be fully excluded.

The migration matrix shows a pretty high correlation between the gamma prediction values and the true values from CORSIKA.

The SVD unfolding and the Poisson-likelihood unfolding show very similar results. The number of events per energy differs slightly between the two methods. But these differences lie in the range of the calculated standard deviations. Therefore, none of the methods seems to be better suited for this problem than the other.

Because of this, the calculated flux also shows very similar results for both methods. The values of the flux, according to the respective energies, are nearly identical. Only the flux values for the highest energy differ slightly, but are still in the range of the calculated standard deviation.

A nearly linear correlation between the flux and the energy is shown. For higher energy values, the flux values are lower.

This is also shown in the two fitting functions from HEGRA and MAGIC. The flux values of the likelihood unfolding done in this experiment are mostly compatible with the two functions. Only the first value, which is lower than 1 TeV, differs a bit from the functions. Nevertheless, the errorbar of this value reaches nearly the two functions.

Generally, the function from HEGRA seems to describe the calculated values from this experiment slightly better.

\section{Discussion}
\label{sec:Diskussion}
The selected data possesses high fluctuations when compared with the expected background.
If the errors of the data is calculated poisson-like, these fluctuations are still in the standard deviation.
This indicates that the difference between Monte Carlo and data are most likely statistical.
Using the full data set instead of just one sample could lower these statistical fluctuations.


The statistical analysis seemed to worked fine and the lower mass limit of $Z^\prime$ is at $\SI{1300}{\giga\electronvolt}$.
Again, using the full dataset could increase this lower limit.

In terms of this lab course it was very helpful in understanding how a more or less complete data analysis at ATLAS is looking like.
Some complications were appearing when facing the languages \texttt{C++} and \texttt{Root} since they were not that common to use.
A small introduction or examples would be very helpful.

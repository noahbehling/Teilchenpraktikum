\section{Theory}\label{sec:Theory}
In this Chapter a brief introduction to the standard model of particle physics (SM) and its important aspects regarding cosmic rays is given.

\subsection{The Standard Model of Particle Physics}
The SM \cite{griffiths} describes the currently known elementary particles that all matter consists of and all fundamental interactions except gravity.
The elementary particles are divided into fermions and bosons.
Fermions have spin $1/2$ and can be further divided into quarks and leptons.
There are six quarks called \textit{up} (u), \textit{down} (d), \textit{charm} (c), \textit{strange} (s), \textit{top} (t) and \textit{bottom} (b).
These fall into three generations
\begin{equation*}
  \binom{u}{d},\binom{c}{s},\binom{t}{b}.
\end{equation*}
The \textit{up-type} quarks u, c and t have an electric charge $Q = \frac{2}{3}$ while the \textit{down-type} quarks d, s and b have an electric charge $Q = - \frac{1}{3}$.
Additionally, quarks come in three different color charges red, green, and blue and their respective anticolours.

Also the six leptons are categorized into three generations consisting of one of the charged leptons electron, muon and tau and their respective left-handed neutrino:
\begin{equation*}
  \binom{e^-}{\nu_e},\binom{\mu^-}{\nu_\mu},\binom{\tau^-}{\nu_\tau}.
\end{equation*}
The electric charge of the charged leptons is $\num{-1}$ while neutrinos are neutral.

Each class of fermions is divided into generations since they show the same physical behaviour, but have a higher mass in comparison to lower generations.\\

Bosons have an integer spin and are distinguished between gauge bosons and the Higgs boson.
Gauge bosons have spin $1$ and are mediators of the fundamental interactions. Photons mediate the electromagnetic interaction and couple to electric charge, gluons the mediators of the strong interaction and couple to color charge and the $W^{\pm}$ and the $Z^0$ bosons mediate the weak interaction.

The Higgs however has a spin $0$ and does not act as a mediator. It is the newest addition to the SM and was discovered 2012 at the LHC\cite{ATLAS:2012yve, CMS:2012qbp}. 

\subsection{Cosmic Rays}
The earth is constantly bombarded with highly energetic particles. These cosmic rays (CR) mostly consist of protons ($\approx \SI{85}{\percent}$) and $\alpha$-particles ($\approx \SI{10}{\percent}$), but also heavy nuclei and electrons can be observed. This radiation can achieve energies up to $10^{20}\,\si{\electronvolt}$ and the energy spectrum can approximately be described by a power law
\begin{equation}
  \frac{\mathrm{d}\Phi}{\mathrm{d}E} = \Phi_0 E^{\gamma},
\end{equation}
where $\gamma \approx - 2.7$ is the spectral index.

Due to the electric charge of the CR particles they are deflected in the galactic and extragalactic magnetic fields resulting in an isotropic flux from all directions.\\

The muons and neutrinos detected by IceCube can come from either atmospheric or astrophysical sources. First can be separated into prompt and conventional muons and neutrinos regarding their energetic spectrum. Conventional muons and neutrinos come from pions and kaons formed when the primary CRs interact with the earth's atmosphere. Due to their comparatively large lifetimes these particles lose energy before decaying into muons and neutrinos resulting in a energy spectrum $\propto E^{-3.7}$. But at high energies also $D$ mesons and $\Lambda_c$ baryons are produced. These have very short lifetimes and therefore do not lose any energy before decaying. This leads to an energy spectrum $\propto E^{-2.7}$.

It is expected that cosmological sources of accelerated hadrons also emit photons and neutrinos. These are not deflected in the galactic and extragalactic magnetic fields due to their lack of electric charge resulting in their arrival directions pointing back to their sources. Due to the small cross section of neutrinos, they are also able to pass through dense dust clouds and other optically dense media which would absorb the emitted photons.

Under the assumption of shock acceleration \cite{Fermi}, a spectral index $\gamma \approx 2$ is expected.

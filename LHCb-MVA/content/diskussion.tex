\section{Discussion}
\label{sec:Diskussion}
In this analysis a search for the rare decay $B^0_s \to \psi(2S)K^0_\mathrm{S}$ was done in data collected by the LHCb experiment. To reduce the combinatorial background in the data a multivariate classifier was trained on $B^0_s \to \psi(2S)K^0_\mathrm{S}$ MC simulation and the USB of the data to distinguish between signal and background. The data also contains events from the decay $B^0_d \to \psi(2S)K^0_\mathrm{S}$ which is kinematically very similar and therefore not rejectable by the classifier.
Therefore a fitting model is applied to the selected data consisting of four Gaussian distributions of which two model one mass peak, as well as an exponential function modelling the surviving combinatorial background. From the fit the number of signal yield can be extracted and by comparison to the background events in the signal region a statistical significance proxy of $\num{4.6}\sigma$ can be obtained.
In the context of the LHC, this is an evidence for the observation of the decay $B^0_s \to \psi(2S)K^0_\mathrm{S}$, because the significance is over $\num{3}\sigma$, but it cannot be classified as an observation since the boundary for an observation is $\num{5}\sigma$.
Nevertheless, this analysis demonstrated the power of multivariate classifiers as a tool to efficiently reject combinatorial background and the importance to understand the examined decay to adjust the fitting model to possible non-reducible backgrounds and allowed to get an idea of the analysis structure used to observe rare decays in the LHCb group.
To further enhance this analysis, new features could be constructed for the classification to reject the combinatorial background even more efficiently. Also, systematic uncertainties would need to be studied.

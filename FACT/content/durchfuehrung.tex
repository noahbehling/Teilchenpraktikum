\section{Datasets}
\label{sec:Durchführung}
For the analysis three different datasets available at \href{https://factdata.app.tu-dortmund.de/fp2/}{https://factdata.app.tu-dortmund.de/fp2/} are being used: 
\begin{itemize}
    \item \texttt{open\_crab\_sample\_d13.hdf5}: reconstructed measurements from $17.7 \, \text{h}$ of observation of the crab nebula with FACT
    \item \texttt{gamma\_test\_d13.hdf5}: reconstructed simulated $\gamma$-events
    \item \texttt{gamma\_corsica\_headers.hdf5}: information about the simulated gamma air showers .
\end{itemize}
These datasets, except for the last one, are the result of the reconstruction of the type of particle, energy of the particle and direction 
of origin with the software \texttt{FACT-Tools} version $1.1.2$ from either simulated or measured data \cite{FACT-tools}. \\
For the simulated data the airshowers and cherenkov production have been simulated with \texttt{CORSIKA}\cite{CORSIKA} while the detector response has been simulated with 
\texttt{CERES}. The resulting data has the same format as the data directly from the telescope except that the initial events are known. 
Three different datasets have been simulated using this method:
\begin{enumerate}
    \item $\gamma$-rays originating froim a point like source observed in the wobble-mode
    \item diffuse $\gamma$-rays originating from random directions in the field of view of the telescope
    \item diffuse protons originating from random directions in the field of view of the telescope.
\end{enumerate}
The dataset of simulated events \texttt{gamma\_test\_d13.hdf5} contains $70 \, \%$ of the simulated events.
A Random Forest Regressor is trained on the dataset as an energy estimater and a Random Forest Classifier is trained to distinguish between diffuse 
$\gamma$-rais and protons.

\section{Strategy}
For the analysis only events that are classified by the Random Forest Classifier as $\gamma$-radiation with a confidence $\geq 0.8$ are chosen. \\
After this a $\theta²$-plot is created with the measured values that have a distance from the on- or off-positions $\theta² \leq 0.025°²$ \\
Next the energy migrationmatrix of the Random Forest Regressor with logarithmically equidistant bins between $500 \, \si{\giga\eV}$ 
and $15 \, \si{\tera\eV}$ is determined. Utilizing the energy migrationmatrix and the background estimate from the wobble-mode 
a Naive SVD-deconvolution and a Poisson-Likelihood deconvolution is performed.\\

Finally the result are compared to the publicised results of the MAGIC- and HERA-collaboration.

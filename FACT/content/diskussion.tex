\section{Discussion}
\label{sec:Diskussion}

The $\theta^2$-plot shows that for low values of $\theta^2$ the events from the on- and off-regions are well distinguishable while at higher values the on-region blends in with the isotropic background of the off-regions. This motivates the cut at $\theta^2 \leq \num{0.025}$. It also shows that most of the on-events are registered at $\theta^2 \approx 0$ while the off-events are distributed isotropically. This is due to the fact that the telescope is pointed directly at the Crab Nebula. This is also reflected in the high significance of detection $S = \num{26.28}$.

The results of the Naive SVD and the Poisson-Likelihood unfolding are in agreement with each other within one standard deviation. Therefore none of the two methods seems to be better suited for the given problem.

The same is true for the flux. With both methods a nearly linear decending correlation of flux and energy is found. This finding is in agreement with the measurements of the HEGRA and MAGIC experiments within one standard deviation above $\SI{1}{\tera\eV}$. Below that energy a slight deviation is observed in this analysis from the other two, but one has to take into account that only one standard deviation is shown in the plots and the measurements are in agreement with each other for most of the studied energy range. Therefore the deviation at low energies is not alarming.

\section{Analysis strategy}
\label{sec:Durchführung}
In this analysis a search for $t\bar{t}$ resonances is done. Since the \textit{all-hadronic} channel has a very large background contribution and the \textit{dilepton} channel has a very small branching fraction, only the \textit{lepton+jets} channel is considered in this analysis.
To identify events with a neutrino, a minimum missing transverse momentum is required, because the momentum the undetectable neutrino carries cannot be inferred otherwise. To distinguish between $b$ and light-hadron-jets it is exploited that $B$ mesons have a relatively long lifetime, resulting in a displaced secondary vertex in the pixel detector. An event selection is applied using these criteria among others to increase the signal-to-background ratio. To model the background contribution Monte Carlo (MC) simulations are used. To check the validity of the MC simulations these are compared to the data. In the end a final discriminant is chosen and used in a statistical analysis to determine the amount of signal on top of the background distribution.

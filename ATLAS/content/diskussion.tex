\section{Discussion}
\label{sec:Diskussion}
In this analysis, a search was conducted for a hypothetical $Z^\prime$. In the process multiple fundamental distributions of events 
%and particles included in events 
have been visualised and compared with the expected simulated background from the Standard Model processes. It was found that 
%there is overall an excess of events in the experiment, which was not immediately interpreted as a signal of a new paricle. 
there is an overall excess of events in datacompared to the simulated events, which cannot immediately be interpreted as a signal of a 
new particle, but rater as imperfections in the MC simulations.
One other interpretation is
that the excess could be combinatorial background that was not removed by the cut. To account for that the analysis could be done again with more restrictive cuts.
Finally, a statistical analysis of 
%the discrepancy between 
simulated events and data was conducted and no evidence for a signal was observed. 
Since no signal was observed $95\%$ CL. on the cross section of the hypothetical $Z^\prime$ have been determined for the 
considered masses.
A mass
range from $\approx 410 \, \si{\giga\eV}$ to $\approx 1800 \, \si{\giga\eV}$ has been excluded for a hypothetical $Z^\prime$.\\
Results of the analysis could most likely be improved by optimising the used cuts and utilising the muon samples in addition to the electron samples. \\
%The lab course was very helpful in understanding how the ATLAS data is stored and can be used for analysis. Also it was very interesting to see how
%an advanced topic like a new particle could be approached with relatively simple tools like the representation of the data and cuts on the data. An introduction 
%in how to manipulate the data structures in root would have been helpful in hindsight.

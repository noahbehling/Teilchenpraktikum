\section{Discussion}
\label{sec:Diskussion}
In this analysis a search was conducted for a hypothetical $Z^\prime$. In the process multiple fundamental distributions of events and particles included 
in events have been visualised and compared with the expected simulated background from the Standard Model. It was found that there is overll an excess 
of events in the experiment, which was not immediately interpreted as a signal of a new paricle. One possible interpretation of the excess events is 
that they could be combinatorial background that was not removed by the cut. To account for that the analysis could be done again with more restrictive cuts. 
Finally a statistical analysis of the descrpency between simulated events and data was conducted and no evidence for a signal was observed. In the end a mass 
range from $\approx 410 \, \si{\giga\eV}$ to $\approx 1800 \, \si{\giga\eV}$ has been excluded for a hypothetical $Z^\prime$.\\
Results of the analysis could most likely improved by optimising the used cuts and utilizing the muon samples in addition to the electron samples. \\
The lab course was very helpful in understanding how the ATLAS data is stored and can be used for analysis. Also it was very iunteresting to see how 
an advanced topic like a new particle could be approached with a relatively simple tools like the representation of the data and cuts on the data. An introduction 
in how to manipulate the data structures in root would have been helpful in hindsight.